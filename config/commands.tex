% common commands

\ifthenelse{\equal{\TwoSide}{true}}
{
    % TwoSide settings
    % Use default `\cleardoublepage'
}
{
    % OneSide settings
    \renewcommand{\cleardoublepage}{\clearpage}
}

\newcommand{\bful}[1]{{\bfseries\uline{#1}}}

% Commands to input body pages

% undergraduate
\newcommand{\inputpage}[1]{\inputCommon{page/undergraduate/#1}}
\newcommand{\inputbody}[1]{\inputCommon{body/undergraduate/#1}}

% TODO: how to switch to \section* without breaking current formats?
\newcommand{\chapternonum}[1]
{
    \cleardoublepage
    \phantomsection
    \addcontentsline{toc}{chapter}{#1}
    \begin{center}
        \bfseries \zihao{3} #1	
    \end{center}	
    \setcounter{section}{0}
}

\newcommand{\sectionnonum}[2][openright]
{

    \ifthenelse{\equal{\TwoSide}{true}}{
        \ifthenelse{\equal{#1}{openright}}
            {\cleardoublepage}
            {\ifthenelse{\equal{#1}{openany}}{\clearpage}{}}
    }{
        \ifthenelse{\equal{#1}{openright}}
            {\clearpage}
            {\ifthenelse{\equal{#1}{openany}}{\clearpage}{}}
    }
    \phantomsection
    \addcontentsline{toc}{section}{#2}
    \begin{center}
        \bfseries \zihao{3} #2	
    \end{center}	
    \setcounter{subsection}{0}
}


\newcommand{\signature}[1]
{
    \begin{flushright}
        \bfseries \zihao{-4}
        #1 \underline{\multido{}{5}{\quad}} \\
        \quad 年 \quad 月 \quad 日
    \end{flushright}
}

\DeclareDocumentCommand{\finaleval}{O{~} O{~} O{~} O{~} O{~}}
{
    \begin{table}[H]
        \centering \bfseries
        \begin{tabularx}{\textwidth}{|>{\fangsong}c
                                     |>{\fangsong}X<{\centering}
                                     |>{\fangsong}X<{\centering}
                                     |>{\fangsong}X<{\centering}
                                     |>{\fangsong}X<{\centering}
                                     |>{\fangsong}c|}
            \hline
            \makecell{成绩\\比例}
            & \makecell{\ifthenelse{\equal{\Type}{thesis}}{文献综述}{中期报告} \\(10\%)}
            & \makecell{开题报告\\(15\%)}
            & \makecell{外文翻译\\(5\%)}
            & \ifthenelse{\equal{\Type}{thesis}}{毕业论文质量及答辩(70\%)}{毕业设计质量及答辩(70\%)}
            & \makecell{总评\\成绩} \\

            \hline
            \multirow{2}*{分值}
            & \multirow{2}*{\zihao{4}#1}
            & \multirow{2}*{\zihao{4}#2}
            & \multirow{2}*{\zihao{4}#3}
            & \multirow{2}*{\zihao{4}#4}
            & \multirow{2}*{\zihao{4}#5} \\

            ~ & ~ & ~ & ~ & ~ & ~ \\
            \hline
        \end{tabularx}
    \end{table}
}

% `design` commands
\DeclareDocumentCommand{\designproposaleval}{O{~} O{~}}
{
    \begin{flushright}
        \begin{tabular}{| >{\fangsong \zihao{4}}c
                        | >{\fangsong \zihao{5}}c
                        | >{\fangsong \zihao{5}}c |}
            \hline
            \multirow{2}*{成绩比例}
            & 开题报告
            & 外文翻译 \\

            ~
            & 占(15\%)
            & 占(5\%) \\

            \hline

            \multirow{2}*{分值}
            & \multirow{2}*{\zihao{4}#1}
            & \multirow{2}*{\zihao{4}#2} \\
            
            ~
            & ~
            & ~ \\
            \hline
        \end{tabular}
    \end{flushright}
}

\DeclareDocumentCommand{\designmidcheckeval}{O{~}}
{
    \begin{flushright}
        \begin{tabular}{| >{\fangsong \zihao{4}}c
                        | >{\fangsong \zihao{5}}c |}
            \hline
            \multirow{2}*{成绩比例}
            & 中期报告 \\

            ~
            & (10\%) \\

            \hline

            \multirow{2}*{分值}
            & \multirow{2}*{\zihao{4}#1} \\

            ~
            & ~ \\
            \hline
        \end{tabular}
    \end{flushright}
}

% `thesis` commands
\DeclareDocumentCommand{\thesisproposaleval}{O{~} O{~} O{~}}
{
    \begin{flushright}
        \begin{tabular}{| >{\fangsong \zihao{4}}c
                        | >{\fangsong \zihao{5}}c
                        | >{\fangsong \zihao{5}}c
                        | >{\fangsong \zihao{5}}c |}
            \hline
            \multirow{2}*{成绩比例}
            & 文献综述
            & 开题报告
            & 外文翻译 \\

            ~
            & 占(10\%)
            & 占(15\%)
            & 占(5\%) \\

            \hline

            \multirow{2}*{分值}
            & \multirow{2}*{\zihao{4}#1}
            & \multirow{2}*{\zihao{4}#2}
            & \multirow{2}*{\zihao{4}#3} \\

            ~
            & ~
            & ~
            & ~ \\
            \hline
        \end{tabular}
    \end{flushright}
}


% From: https://tex.stackexchange.com/questions/395856/switching-tocdepth-in-the-middle-of-a-document
\newcommand{\changelocaltocdepth}[1]{%
  \addtocontents{toc}{\protect\setcounter{tocdepth}{#1}}%
  \setcounter{tocdepth}{#1}%
}